%
% LaTeX template reporte
%
\documentclass[]{article}
\usepackage[paperheight=27.94cm,paperwidth=21.59cm,bindingoffset=0in,left=2.5cm,right=2.0cm, top=2.5cm,bottom=2.5cm, headheight=200pt, headsep=1.0\baselineskip]{geometry}
\usepackage{graphicx,lastpage}
\usepackage{upgreek}
\usepackage{censor}
\usepackage[spanish,es-tabla]{babel}
\usepackage{pdfpages}
\usepackage{tabularx}
\usepackage{graphicx}
\usepackage{adjustbox}
\usepackage{xcolor}
\usepackage{colortbl}
\usepackage{rotating}
\usepackage{animate}
\usepackage{multirow}
\usepackage[utf8]{inputenc}
\renewcommand{\tablename}{Tabla}
\usepackage{fancyhdr}
\usepackage{movie15}
\pagestyle{fancy}
\fancyhf{}
\renewcommand{\footrulewidth}{0.4pt}
\lhead[\leftmark]{Criptografía.}
\rhead[Taller de Redes: Tarea 5]{\rightmark}
\lfoot[\thepage]{}
\rfoot[]{\thepage}
\usepackage{moreverb}
\usepackage{enumitem}
\usepackage{float}
\usepackage{minted}
\usepackage{hyperref}
\usepackage{xcolor}

\definecolor{mygreen}{RGB}{0,128,0}
\hypersetup{
    colorlinks=true,
    linkcolor=black,
    filecolor=magenta,      
    urlcolor=mygreen,
}
\usepackage{listings}
\lstset{basicstyle=\ttfamily,
  showstringspaces=false,
  commentstyle=\color{red},
  keywordstyle=\color{blue}
}



\begin{document}
%------------------Portada--------------
    \begin{titlepage}
        \centering
        \vspace{1cm}
        {\scshape\Large Universidad Diego Portales \par}
        \vspace{2cm}
        {\scshape\Huge Tarea 1: PASSWORD\par}
        \vspace{1cm}
        \vspace{1cm}
        {\itshape\Large Profesor: Víctor Manriquez \par}
        \vfill
        {\Large Integrantes: \par}
        {\Large Diego Carrillo \par}
        \vfill
        {\Large 15 de Mayo 2022 \par}
    \end{titlepage}
%-----------------------------------------------

\tableofcontents
 \newpage
 \section{Introducción}
 En el presente laboratorio, se enfoca principalmente en auditar un sitio web, en donde se pondrá a prueba la seguridad e integrad de este mismo sitio. 
 \\\\
 Poniendo a prueba, la capacidad, de cada pagina, para detectar ataques de fuerza bruta, o bien, procesos automatizados.
 \\\\
 Estos procesos automatizados, fueron generados para contraseñas que se encontraron, para este mismo laboratorio, mediante el uso de Dorks. 
\section{Desarrollo}
    A continuación, se mostraran los hitos que componen de este laboratorio, en conjunto con vídeos,
    fotos y código correspondiente para cada caso. 

\subsection{Hito I}

    Para empezar esta actividad, se debió de buscar filtraciones de datos en paginas Chilenas, mediante el uso de dorks.
     Estas filtraciones,deberían de poder contener como mínimo un correo y contraseña para cada usuario.      
    \\\\
    Durante el desarrollo de este hito, surgió una gran cantidad de inconvenientes, debido a que, si bien, se poseia un conocimiento
    sobre el uso de dorks, pero era limitado hasta un cierto punto.Y por otro lado, los sitios en donde se encontraban las filtraciones
    tambien eran limitados.
    \\\\
    Se nos recomendó tener como referencia la pagina de pastebin, el cual es un sitio en donde la comunidad sube sus aportes y datos que han
    obtenido mediante filtraciones a gran escala o simplemente datos que obtienen de manera individual.
    \\\\
    Como los datos que se fueran encontrando, debian de irse rellenando en un excel; provocó que exista competitividad dentro de los 
    integrantes del curos, ya que, los links que se obtuvieron, se repetian, lo que negó el uso de los datos encontrados. Este hito,
    el estudiante lo realizo, aproximadamente, unas 5 veces y fue el que mas tiempo tomó debido a la escases de datos. 
    \\\\
    El Dork utilizado para encontrar las credenciales utilizadas fue el siguiente:
    \\\\
    \begin{center}
        \textbf{intext:"LulzSecAR" site:'pastebin.com'}
    \end{center}
    \\\\
    Fue creado para ser utilizado en el buscado de Google, y basicamante, se basa en buscar dentro de la pagina pastebin.com a un usuario, el cual, es conocido por que sus publicaciones o
    comentarios son filtraciones, entonces, se decide ir pos las publicaciones de este usuario para hacer una busqueda que garantice
    un resultado. 

    \\\\
    Posteriormente, 

    

\newpage

\subsection{Hito II}

\subsection{Hito III}

\subsection{Hito IV}



\bibitem{Definicion SCADA} 
Definicion SCADA,
\\\url{https://www.redeszone.net/2017/01/05/conoce-este-motor-busqueda-ciberespacio-podras-localizar-host-facilmente/}
\\



\end{document}
